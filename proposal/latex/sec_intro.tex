%!TEX root = main.tex


Deep reinforcement learning has been attracted widely interests from both
industry and academia. The reason is that the combination of deep learning and
reinforcement learning has shown the effectiveness on lots of applications
and more surprisingly, the good performances can be achieved without extensive
 problem-specific engineering~\cite{mnih2015human,silver2016mastering}.

In the center of most reinforcement learning algorithms is how the agent interacts
with the environment. At each step, the agent takes the corresponding \emph{action}
based on the \emph{observation} and \emph{reward} received, which determine a internal
agent \emph{state} that is used by reinforcement learning algorithms. The mapping function
from agent state to an action is designed to maximize the cumulative reward~\cite{sutton1998introduction}. 

Lots of advanced algorithms were proposed to learn the function that project
state to an action. One track is to formulate an action-value function $Q(s,a;
\theta)$ and always choose the action gives you the maximum Q-value, which is
named as value-based model-free reinforcement learning method. In contrast to
above methods, policy-based model-free methods directly parameterize the
policy ($\pi(a|s; \theta)$) and update the parameters $\theta$ by performing
policy gradient. In this project, we will first review the existing techniques in 
deep reinforcement learning and proposed a method to outperform the state-of-arts 
on a specific problem, such as Atari 2600 on OpenAI Gym.


%% Open AI
OpenAI Gym~\cite{brockman2016openai} is a toolkit for reinforcement learning
research. It includes a growing collection of benchmark problems that expose a
common interface, and a website where people can share their results and
compare the performance of algorithms.
